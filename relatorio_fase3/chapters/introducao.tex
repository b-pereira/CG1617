%---------------------------------------------------------------------------------------------------------------%
\section*{Introdução\markboth{\MakeUppercase{Introdução}}{}}
\addcontentsline{toc}{section}{Introdução}
\label{intro}


Para este projeto é requerido que se criem cenas hierárquicas com transformações
geométricas descritas num ficheiro XML.\ Cada cena é uma árvore onde cada nodo
contem um conjunto de transformações geométricas e conjuntos de modelos com os
nomes de ficheiros onde estão vértices para a criação da cena. O resultado final
é  \emph{renderização} de um modelo do sistema solar.  

Adicionalmente, os objetivos passam por criar uma função de leitura dos dados
a partir de ficheiros, com a criação de uma estrutura ou estruturas de
estruturas para armazenamento dos dados, sendo este processo feito apenas uma
vez. Também é necessário definir primitivas gráficas e um processo de leitura
das estruturas para ser incluído no processo de \emph{rendering}.  


Este relatório está divido em duas secções: a secção que documenta o processo de
criação e programação de algoritmos para criação de objetos para o sistema
solar, no programa \texttt{Engine}, e uma segunda secção que documenta
o processo de leitura dos dados gerados pelo primeiro programa, estrutura
utilizadas para guardar esse dados, e processo de \emph{rendering} iterando
sobre as estruturas de dados e aplicando transformações, de forma a construir
o sistema solar.



