%---------------------------------------------------------------------------------------------------------------%
\section*{Introdução\markboth{\MakeUppercase{Introdução}}{}}
\addcontentsline{toc}{section}{Introdução}
\label{intro}


Para esta fase deste projecto é requerido o desenvolvimento de um novo tipo de modelo geométrico baseado em superfícies de Bezier, este novo modelo vai desempenhar o papel de um asterioide no sistema solar já desenvolvido na segunda fase. Este asteroide vai ter por si só uma órbita sua, consistente com a de um asteróide real, isto é, embora elíptica, a diferença entre a distância ao sol no momento da órbita de maior proximidade e a de maior distância irá ser maior do que qualquer planeta, dando assim um toque de realismo.

Adicionalmente, outra novidade irá ser as alterações relativas à translação e rotação dos planetas. As órbitas dos planetas agora irão ser sobre curvas Catmull-Rom, e as tranlações referentes a cada planeta deverão ser interpretadas do ficheiro XML em função de um ``time'' e dos seus pontos de controlo necessários para a curvatura Catmull-Rom. Quanto à rotação, nesta fase é subtituída a leitura pelo seu ``angle'' e passa ser igualmente por um ``time''. 

Finalmente, nesta fase todos os modelos apresentados irão ser desenhados com VBOs, em oposição ao modo imediato usado nas fases prévias.

Todas estas funcionalidades irão ser faladas em detalhe mais abaixo, dando ao leitor uma breve introdução teórica sobre cada tópico de modo a serem melhor compreendidas as decisões tomadas pelo grupo.