%---------------------------------------------------------------------------------------------------------------%
\section*{Conclusão\markboth{\MakeUppercase{Conclusão}}{}}
\addcontentsline{toc}{section}{Conclusão}
\label{concl}

Em suma, pode-se afirmar que a maioria dos objetivos foram cumpridos, ou seja
existe um modelo do sistema solar que pode ser \emph{renderizado} com primitivas
gráficas personalizadas, a partir da leitura de um ficheiro XML, cujo os dados
são carregados numa estrutura em memória. Além do mais, existe a possibilidade
de aplicar transformações a essa primitiva, tais como rotações, translações
e escalas.


Com efeito, na primeira secção, relativa ao programa \texttt{Generator} que
calcula os vértices de triângulos, foram implementas duas funções para desenhar
esferas e discos com determinada espessura, para serem utilizados na construção
do sistema solar. Adicionalmente, nesta secção são apresentados os algoritmos
para o cálculos dos vértices, bem como diagramas explicativos, como também
resultados da implementação desse algoritmos, e por último, fórmulas.\

Na segunda secção, relativa ao programa \texttt{Engine}, que \emph{renderiza} os
vértices guardados em ficheiros, foram implementadas rotinas para a leitura de
dados, sendo, este processo, feito apenas uma vez, guardando os dados lidos em
estruturas de dados adequadas. De igual modo criou-se um algoritmo para iteração
nessas estruturas de dados, durante o processo de \emph{rendering}.
À semelhança, com a secção anterior, este secção apresenta resultados,
algoritmos e diagramas.


No entanto ficaram duas coias por fazer: otimização da função de leitura
(\emph{parsing} de linhas), uma vez que este processo tem várias iterações em
cada resultado da linha obtida, e a implementação da câmara em primeira pessoa
não foi possível, uma vez que, não houve tempo para amadurecer conhecimentos.



