%---------------------------------------------------------------------------------------------------------------%
\section*{Conclusão\markboth{\MakeUppercase{Conclusão}}{}}
\addcontentsline{toc}{section}{Conclusão}
\label{concl}




Para concluir este relatório, pode-se dizer que todos os objetivos a cumprir foram concluídos sem grandes precalços nem atrasos. Isto é, o modelo bule a desenvolver com superfícies Bezier, as órbitas serem feitas sobre curvas Catmull-Rom e desenhar os modelos com VBOs foi tudo conseguido com sucesso. 

Quanto ao bule desenvolvido que presta o papel de asteróide no sistema solar, apesar de ser o que causou mais receio numa primeira fase, foi algo que surpreendemente não causou grande empasse. O resultado obtido relativamente a esta matéria ficou o que seria expectável, e o grupo espera que a explicação sobre o processo de desenvolvimento tenha sido ilucidativo e conciso para que até o leitor que não esteja confortável com este tema, lhe seja feita uma breve introdução à teoria por trás.

Relativamente às órbitas sobre o Catmull-rom optou-se por usar por cada órbita 16 pontos de controlo de modo a que o produto final fosse uma curva suave. Nesta secção a maior dificuldade foi chegar a um consentimento de como se ia fazer a lua orbitas sobre o seu planete correspondente, mas como foi explicado em cima, também se ultrapassou esta dificuldade.

Quanto à mudança do modo de desenho dos modelos, passando agora os modelos a serem desenhados com VBOs, também foi algo que necessitou de consiliação de conhecimentos com estudo antes de se conseguir abordar esta matéria, no entanto, aqui a aula prática onde VBOs foi lecionado prestou um papel fulcral.