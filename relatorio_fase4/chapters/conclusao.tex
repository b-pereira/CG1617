%---------------------------------------------------------------------------------------------------------------%
\section*{Conclusão\markboth{\MakeUppercase{Conclusão}}{}}
\addcontentsline{toc}{section}{Conclusão}
\label{concl}

Em suma, podemos concluir que o projeto foi um sucesso parcial, dado que não foi
possível iluminar os topos do bule, e que possivelmente poderiam ter sido
efetuados mais testes. Não foram geradas coordenadas de textura para a caixa,
cone e plano, nem vetores normais destas geometrias. No entanto, ficaram
implementadas os vários tipos de luzes, com as suas propriedades de cor, bem
como as propriedades materiais da geometria, assim como a aplicação de texturas.


Como trabalho futuro, sugere-se implementar o cálculo dos vetores normais
e coordenadas de textura para as geometrias que faltam, bem tentar o cálculo de
vetores normais através de interpolação para o bule, tentando diminuir o tamanho
de cada triângulo, para introduzir o efeito MACH ou usar diretamente o modelo
\emph{flat} sabendo das desvantagens de quaisquer um destes métodos. 


