\section{Motor}

Nesta secção pretende-se descrever o funcionamento do Motor \emph{Engine},
e para tal pretende-se abordar vários aspetos desde as estruturas de dados
auxiliares, estruturas de dados para os diferentes tipos de objetos que compõem
o motor, descrição do processo de leitura e do processo de \emph{rendering}.
Para este projeto usaram-se \emph{vertex array objets} para o \emph{rendering}
das geometrias, tanto vértices, vetores normais, como também coordenadas de
textura.


\subsection{Estruturas de dados auxiliares}


Para poder manipular dos dados, tanto dos vértices, normais, coordenadas de
textura, como também informação sobre os \emph{vertex array objets --- VBOs},
luzes com as suas propriedades, identificadores de texturas e ainda a raiz da
estrutura de dados para armazenar informação do ficheiro XML, criou-se um tipo
de dados \emph{Models}. 

Este tipo de dados é composto por vários \texttt{vector}, dos quais um para os
vértices, outro para os vetores normais e outro para coordenadas de textura.
Note-se que estes vetores contêm os identificadores para a inicialização
e \emph{rendering} dos VBO's, sendo que a informação para o \emph{rendering} dos
VBO's se encontra num tipo de dados homónimo, com um índice para os
\texttt{vector} para e o total de vértices. De salientar que o valor de índice
e do número de vértices é partilhado pelos diferentes VBO's, de vértices,
normais e texturas, uma vez que o número de ocorrência de vértices, normais
e pontos da textura são os mesmos, e consequentemente, o acesso por índice
a cada um dos \emph{vector} de identificadores é igual. Porém, na função de
inicialização, no preenchimento do \emph{buffer} com os valores para cada tipo
de objeto foi multiplicado o número de componentes (coordenadas) de cada
vértices. Assim para normais e vértices o número de vértices foi multiplicado
por três, e para as coordenadas das textura, o número de vértices foi
multiplicado por 2.

Para cada ficheiro com informação sobre a geometria, existe um e um só
\texttt{VBO} (o tipo de dados), uma vez que a mesma geometria pode ser usadas
várias vezes. Assim existe uma tabela com o nome do ficheiro como chave, tendo
associado o tipo \texttt{VBO} a essa chave. Para as texturas, a estrutura
é similar, só que para cada nome do ficheiro da imagem está associado um número
inteiro não negativo para o identificador da textura. 

Relativamente, à informação de luzes e suas componentes existe um vetor do tipo
\texttt{Light*}, cujo o tipo contém informação sobre as luzes, e posteriormente
neste documento será melhor documentado. Ainda para as luzes, dado que o OpenGL,
na definição de propriedades materiais de cada geometria, essa propriedade
é partilhada de forma global até ser novamente alterada, criou-se um vetor do
tipo \texttt{Materials*} para armazenamento dos valores por defeito da
propriedades de materiais. Assim, após a definição de uma propriedade de uma
material de uma geometria, este vetor é percorrido aplicando as propriedades
materiais por defeito.

Por último, ainda existe uma variável do tipo inteiro par controlo da atribuição
do índice do identificador do VBO's, e raiz para a estrutura de dados com
a informação me memória do ficheiro XML.\ 


\subsection{Estruturas de dados --- classes}

As classes definidas neste projeto servem o propósito de abstrair os conceitos
de computação gráfica como transformações geométricas, definição de luzes e suas
propriedades, bem como classes auxiliares para representação de tipos
geométricos como pontos em 3 dimensões e 4 dimensões e valores RGBA (\emph{red,
green, blue, alpha}).


Para definir a componente de luz material de cada geometria, para além das
normais é necessário definir as quatro características materiais de reflexão da
luz na geometria, tanto componente físicas reais (componente difusa
e especular), como virtuais (ambiente e emissiva). As componentes ambiente
e emissiva são virtuais, uma vez que, não é possível representar a luz ambiente
de uma cena, nem se um objeto emite luz (componente emissiva) e, como tal, não
dependem dos valores dos vetores normais à superfície da geometria.

Para todas as características dos materiais foram definidas classes, numa
hierarquia, e à exceção da componente especular, apenas aplicam um conjunto de
valores representativo dos valores RGBA.\ No entanto, o contexto da sua aplicação
é diferente, uma vez que é invocado a função \texttt{glMaterialfv} com
a respetiva componente (\texttt{GL\_EMISSIVE}, \texttt{GL\_DIFFUSE},
\texttt{GL\_AMBIENT} e \texttt{GL\_SPECULAR}) e é aplicado aquando da invocação
do método \texttt{applyProperties} ou a partir da superclasse \emph{Materials}
ou partir de uma instância da própria classe. A componente especular possui
ainda o brilho da componente especular (\emph{shininess}).A Figura~\ref{ig:ssec2:mat}
apresenta parte da hierarquia de classes mencionada.


\begin{center} 	
\includegraphics[width=\textwidth,height=\textheight,keepaspectratio]{resources/material.png}
\captionsetup{type=figure, width=0.8\linewidth}
\caption{Hierarquia de classes de \emph{Materials}} 
\label{fig:ssec2:mat} 
\end{center}

Para definir as luzes, podem ser definidas até um conjunto de 8 luzes (assumindo
uma abordagem de luzes fixas ao espaço global), podendo estas luzes posicionais 
(\texttt{FixedLight}), direcionais (\texttt{DirectionalLight}) ou
\emph{spotlight} \texttt{Spolight}. Cada um desde tipos de luz pode ter várias
componentes, tais como as geometrias, neste caso apenas sendo três: difusa,
ambiente e especular.

Tal como para as caraterísticas materiais, cada classe possui um conjunto de
valores RGBA, sendo cada característica da luz aplicada no seu contexto com
a função \texttt{glLight}, com o identificador da luz a que estão associadas,
com a componente correspondente (\texttt{GL\_DIFFUSE}, \texttt{GL\_AMBIENT}
e \texttt{GL\_SPECULAR}) através da superclasse \texttt{LightProperty} ou duma
instância da própria classe, à semelhança da hierarquia anterior. Note-se que,
aqui não faz sentido colocar o brilho da reflexão.


\begin{center} 	
\includegraphics[width=\textwidth,height=\textheight,keepaspectratio]{resources/lightProperties.png}
\captionsetup{type=figure, width=0.8\linewidth}
\caption{Hierarquia de classes de \emph{LightProperty}} 
\label{fig:ssec2:props} 
\end{center}

Relativamente ao tipo de luzes, estes podem ser de três tipos: direcional,
posicional e \emph{spotlight}. Cada um deste tipos possui um número variado de
parâmetros, que caraterizam cada tipo de luz. 

A luz direcional apenas é parametrizada pelo identificador da luz e pela direção
da luz, sendo 3 coordenadas do referencial ortonormado, mais a coordenada $w$
(coordenadas homogéneas), com o valor 0, o que representa um vetor. Uma vez que
uma luz direcional pode ser considerada com uma luz infinitamente longe, esta
não possui parâmetros de atenuação, no entanto, possui as propriedades da luz
mencionadas atrás.

Relativamente à luz posicional, à semelhança da luz direcional, para além do
identificador da luz, recebe as coordenadas homogêneas, no entanto a coordenada
$w$ tem um valor de 1, dado que é um ponto. Nesta luz já se podem definir
valores de atenuação, sendo estes valor da atenuação constante, linear ou
quadrática. O inverso da soma deste valores, sendo o valor de atenuação
quadrática multiplicado pelo quadrado da distância entre o ponto de luz
e o vértice de uma geometria e valor da atenuação linear multiplicado pela
distância, é o fator de atenuação. Esta relação está
representada na Equação~\ref{eq:atenuation}.

\begin{equation}
	fator de atenuação = \frac{1}{k_c + k_{l}d + k_{q}d^2}
\label{eq:atenuation}
\end{equation}

Na Equação~\ref{eq:atenuation} $k_c$ é a atenuação constante, $k_l$
é a atenuação linear, $k_q$ é a atenuação quadrática e $d$ é a distância entre
a posição da luz e o vértice da geometria. No OpenGL a característica da
atenuação, pode ser definida pela função \texttt{glLightf}, com o identificador
da luz, podendo ser \texttt{GL\_CONSTANT\_ATENUATION},
\texttt{GL\_LINEAR\_ATENUATION} e \texttt{GL\_QUADRATIC\_ATENUATION}, para
atenuação constante, linear e quadrática respetivamente.

Por último, temos a luz do tipo \emph{spotlight}. Este tipo de luz, para além do
identificador da luz e da posição da mesma, possui uma direção para onde o cone
de luz está apontado e o ângulo de abertura do cone, estando este último a 180º
por defeito, podendo ser alterado para um valor entre 0 e 45º. Adicionalmente,
existe um expoente de concentração de luz. Do mesmo modo, que as anteriores
luzes, possui contribuições do tipo especular, ambiente e difuso.


As coordenadas homogéneas são representadas pela classe \texttt{Point4d}.
À semelhança as outras hierarquias de classes já descritas, existe uma
superclasse (\texttt{Light}) sendo as características de cada tipo de luz,
aplicadas pela função \texttt{applyProperties} no contexto desta superclasse, ou
numa instância de cada uma das subclasses. Cada uma das classes implementa
a função \texttt{glLightsfv}, aplica as contribuições e ativa a luz, no método
\texttt{applyProperties}, nesta ordem. Adicionalmente, existe uma método
\texttt{addProperty}, que adiciona as contribuições de natureza especular,
difusa ou ambiente. A hierarquia de classes pode ser vista em maior detalhe na
Figura~\ref{fig:ssec2:lights}.


\begin{center} 	
\includegraphics[width=\textwidth,height=\textheight,keepaspectratio]{resources/light.png}
\captionsetup{type=figure, width=0.8\linewidth}
\caption{Hierarquia de classes de \emph{Light}} 
\label{fig:ssec2:lights} 
\end{center}


\subsection{\emph{Vertex Array Objects}}

O OpenGL possibilita dois modos de renderização: o modo imediato
e o \emph{vertex buffer objects} (VBOs). Os VBOs permitem um ganho substancial
de performance, uma vez que os dados são logo enviados para a memória da placa
gráfica onde residem, e por isso podem ser renderizados diretamente da placa
gráfica. O modo imediato usa a memória do sistema onde os dados são inseridos
\emph{frame} a \emph{frame}, usando uma API de renderização, o que causa peso
computacional sobre o processador.


Para a utilização das VBO's é necessário recorrer a criação de \emph{arrays} com
os dados para renderizar geometria, com vetores normais para as luzes
e coordenadas de textura, para uma eventual aplicação desta. Com efeito,
é necessário, em primeiro lugar, ativar os \emph{arrays} com os diferentes tipos
de dados e colocar os dados num \emph{buffer object}. Estes \emph{arrays} são
acedidos pelo seu endereço individual da sua localização em memória, sendo então
desenhadas as figuras geométricas dos respetivos conteúdos dos \emph{arrays}.


Note-se que, no OpenGl, qualquer inteiro sem sinal pode ser usado como um
identificador de \emph{buffer objecto}. Estes identificadores podem-se
armazenados numa estrutura, sendo necessário, em seguida, gerar \emph{buffers}
para os vértices, para os vetores normais e coordenadas de textura (um
\emph{buffer} para cada \emph{array} --- vértices, normais e coordenadas de
textura) \texttt{glGenBuffers} e ativar cada \emph{buffer} pelo seu
identificador (\texttt{glBindBuffer}) e preencher o \emph{buffer} com os dados
de cada \emph{array} previamente mencionado.


Para desenhar, a figura geométrica é necessário definir a semântica, ou seja,
definir o \emph{offset} relativo ao inicio do buffer consoante o tipo de dados,
fazer o bind do objeto apropriado para fazer a renderização dos \emph{arrays} de
vértices, normais e coordenadas de textura usando a função adequada
(\texttt{glDrawArrays} ou \texttt{glDrawElements}).De notar que, cada
\emph{bind} é seguido da semântica. A semântica para os vértices é feita usando
a função \texttt{glVertexPointer}, definido o valor 3 o número de elementos do
tipo valor de vírgula flutuante, correspondente às coordenadas de cada vértice.
Relativamente à semântica para os vetores normais, é utilizada a função
\texttt{glNormalPointer}. Nesta função não é necessário definir o número de
coordenadas, uma vez que a própria função já o faz. No entanto, é necessário
definir o tipo, como valor de vírgula flutuante. Por último, para as coordenadas
das texturas é necessário definir a semântica com a função
\texttt{glTexCoordPointer}, onde o número de elementos para cada vértice será
dois, dado que as textura são bidimensionais. Note-se também, que é necessário
fazer o \emph{bind} da textura pelo identificador obtido pela
função \texttt{loadTexture} antes de proceder ao \emph{rendering} da geometria,
sendo invocada logo de seguida o \emph{bind} para identificador 0, para evitar
que continua a desenhar a mesma textura. Este processo é feito na função
\texttt{drawElement}, que será descrito mais à frente. A inicialização dos
\emph{buffers} é feita pela função \texttt{initBuffers} e para o processo de
\emph{rendering} utilizou-se a função \texttt{drawVBO}, que faz o \emph{bind}
e define a semântica, que é aplicada na função \texttt{drawElement} que aplica
outros elementos para o desenho da geometria, como o \emph{bind} da textura
mencionado acima e aplicação de cores dos materiais, \emph{reset} do mesmos.  


\subsection{Descrição do processo de leitura}


A função de leitura identifica \emph{tags} correspondentes às transformações
geométricas, sejam elas animadas ou estáticas, e à medida que se vai lendo
o ficheiro XML, vai se construindo uma árvore n-ária do tipo \texttt{Group},
como demonstra a Figura~\ref{fig:ssec2:strut}. No entanto, existem dois tipos de
elementos do XML com interesse particular neste projeto, que são os
\texttt{models} e as \texttt{lights}.  

\begin{center} 	
\includegraphics[width=\textwidth,height=\textheight,keepaspectratio]{resources/estrutura.png}
\captionsetup{type=figure, width=0.8\linewidth}
\caption{Árvore \emph{n-ária} para armazenamento de grupos}
\label{fig:ssec2:strut} 
\end{center}

\subsubsection{Modelos}

Para o caso de ocorrência da \emph{tag} \texttt{models}, para cada ocorrência de
uma \emph{tag} filha \texttt{model}, a função \texttt{readXMLFromRootElement}
obtém o nome do ficheiro de vértices, normais e coordenadas de textura. Se
a \emph{tag} \texttt{model} também tiver um atributo \texttt{texture}, para cada
propriedade material, cria a propriedade material de acordo com a ocorrência
(\texttt{diffuse}, \texttt{ambient}, \texttt{specular} e \texttt{emissive}).
Note-se que, se alterou o ficheiro XML para conter \emph{tags} desta forma,
dado que:  é mais legível; mais fácil de tratar cada caso na leitura. Em
seguida, é guardado o nome do ficheiro 3d no vetor de nomes de modelos no tipo
\texttt{Group}, bem como o conjunto de propriedades materiais e nome do ficheio
de textura. Note-se que, pode não existir uma ocorrência de uma textura, ou de
propriedades materiais. No entanto, por uma questão de manter a representação
ordenada para ser acedida por índice, caso não haja uma ocorrência de um
ficheiro de textura, é adicionada a \emph{string} vazia ou o vetor vazio.

Caso o nome da textura existir, isto é, não for uma \emph{string} vazia,
é carregada a textura com a função \texttt{loadTexture}, que devolve um
identificador depois de fazer o \emph{bind} da textura, que ocorre na função
\texttt{loadTexture}. O nome do ficheiro da textura e identificador são
guardados num par chave/valor na estrutura \texttt{Models} mencionada no início
do capítulo. Em seguida é invocada a função \texttt{readFile}, que procederá
a leitura do ficheiro 3d, inserirá um par chave/valor na tabela para o efeito em
\texttt{Models}, com o nome do ficheiro 3d e o tipo \texttt{VBO}, que por sua
vez possui o índice dos vetores de identificadores de \emph{buffers} (vértices,
normais e coordenadas de textura) --- um índice para os três vetores, como já
foi explicado ---, e inicializará os \emph{vertex array objects}.      



\subsubsection{Luzes}

Relativamente às luzes, neste modelo são descritas fora de um grupo, no entanto,
a função de leitura permite a flexibilidade de ter luzes dentro de um grupo.
Mesmo assim, pretende-se que as luzes sejam declaradas no inicio do ficheiro do
documento XML, uma vez que os atributos das luzes e seus tipos são armazenados
numa estrutura à parte (\texttt{Models}), para serem invocadas na
\texttt{renderScene} após o \texttt{gluLookAt}, antes de quaisquer
transformações geométricas, de forma as luzes ficarem fixas no espaço global.

Com efeito, para cada ocorrência da \emph{tag} \texttt{lights}, para cada
elemento \texttt{light} filho, se o tipo (\texttt{type}) for do valor
\texttt{DIR}, cria uma luz direcional, se for do valor \texttt{POINT} cria uma
luz posicional e se for do tipo \texttt{SPOT}, cria um \emph{spotlight}. Para
cada uma desta ocorrências, se existirem propriedades da luz, à semelhança da
propriedades materiais, cada componente é criado conforme a ocorrência
(\texttt{diffuse}, \texttt{specular} e \texttt{ambient}). Estas
componentes são adicionadas à luz, no entanto, podem não existir componentes da
luz, e nesse caso nada é adicionado. A luz é adicionada ao vetor de
\texttt{Light*} em \texttt{Models}.  



\subsection{Descrição do ciclo de \emph{rendering}}



Para fazer o \emph{rendering} da estrutura de dados em memória, implementou-se
uma função de travessia da árvore, colocada na função \texttt{renderScene}, após
a função \texttt{glLoadIndentity} e \texttt{gluLookAt}, nesta sequência.

Com efeito, a primeira instrução é a \texttt{glPushMatrix}, uma vez que se
pretende colocar uma matriz para aplicação das transformações no topo da
\emph{stack} de matrizes do \emph{OpenGL}. Em seguida, para cada transformação
contida num \texttt{Group}, é invocada a função \texttt{applyTrasformation}, que
aplica as transformações as transformações geométricas.

Em seguida, é invocada a função \texttt{drawElement}. Esta função especifica a primitiva
para que será criada com os vértices, normais e coordenadas de textura em
memória, usando a função \texttt{drawVBO}. Adicionalmente, itera pelas as
estruturas de \texttt{Group} (vetores de nomes de ficheiros 3d, vetor de vetor
de \texttt{Materials*}) e procura pela chave do nome do ficheiro, 3d ou textura,
o valor do tipo \texttt{VBO} e identificador da textura, respetivamente, nas
tabelas em \texttt{Models}. Caso o nome do ficheiro 3d existir é que invoca,
a função \texttt{drawVBO}, e antes disso, verifica se o nome da textura existe,
e só se existir é que faz o \emph{bind} da textura. A primeira estrutura a ser
iterada é o vetor de vetores de \emph{Materials*}, sendo que para cada valor de
vetor de \emph{Materials*}, as propriedades são aplicadas conforme o contexto,
pela a invocação do método da superclasse \texttt{applyProperties}. Após
o \emph{rendering} dos \emph{vertex array objects}, todas as propriedades
materiais por defeito do OpenGL são aplicadas, para evitar que propriedades de
diferentes geometrias, sejam aplicadas de forma transversal, até que sejam
alteradas novamente, que a aplicação dos valores por defeito faz.   

Como já foi mencionado, a aplicação das luzes acontece na \texttt{renderScene},
após o \texttt{gluLookAt} e antes da função \texttt{traverseTree} que possui
a descrição do modelo XML em memória. Para a aplicação das luzes aplica-se
o mesmo principio de utilização da hierarquia de classes que se usou até aqui,
e para cada ocorrência de uma instância de \texttt{Light*} no vetor em
\texttt{Models} é invocada a função \texttt{applyProperties} de \texttt{Light},
sendo a mesma função aplicada, no seu devido contexto nas subclasses.


