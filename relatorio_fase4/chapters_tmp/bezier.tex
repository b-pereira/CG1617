\section{Bezier}

Uma curva bezier pode ser definida por um qualquer numero de pontos, pontos estes chamados pontos de controlo da curva. Transformações como translação e rotação podem ser aplicadas na curva manipulando estes pontos. 

O algortimo de Casteljau oferece uma construção geométrica de onde se pode dividir uma curva de Bezier em duas curvas Bezier num parâmetro arbitrário t [0,1]. Como se pode observar:

\begin{center}
 	
 	\includegraphics[scale=0.7,keepaspectratio]{resources/casteljou.png}
 	\captionsetup{type=figure, width=0.8\linewidth}
	\caption{Algoritmo geométrico casteljou.png}
\label{fig:ssec1:diagram:plane:to:sphere} 
\end{center}

Fazendo os cálculos manualmente 
\begin{center}
 	
 	\includegraphics[scale=0.7,keepaspectratio]{resources/casteljeau2.png}
 	\captionsetup{type=figure, width=0.8\linewidth}
	\caption{Representação da árvore casteljeau}
\label{fig:ssec1:diagram:plane:to:sphere} 
\end{center}

Para demonstrar estras tranformações segue-se o exemplo de uma curva de bezier cúbica, isto é, uma curva definida por 4 pontos de controlo:

\begin{center}
 	
 	\includegraphics[width=\textwidth,height=\textheight,keepaspectratio]{resources/exemplos1Bezier.png}
 	\captionsetup{type=figure, width=0.8\linewidth}
	\caption{Curvas exemplo de Bezier}
\label{fig:ssec1:diagram:plane:to:sphere} 
\end{center}

Como se pode observar cada uma das curvas observadas possui 4 pontos de controlo, P0, P1, P2 e P3. A figura demonstra alguma das formas que uma curva de Bezier pode tomar. 
As curvas de Bezier, independentemente do número de pontos de controlo que possam ter, são sempre definidas pela seguinte fórmula:
\begin{equation}
B(t)=\sum_{k=0}^{n}B_{n,k}(t)P_{k}
\end{equation}

onde n corresponde a (nº pontos de controlo - 1), e t é o polinómio Bernstein que tem sempre um valor positivo em [0,1] e o seu somatório é sempre igual a 1.




Com 4 pontos de controlo, uma curva de Bezier diz-se cúbica e pode ser definida pela seguinte formula:




\begin{equation}
B(t)=\sum_{k=0}^{3}B_{3,k}(t)P_{k}
\end{equation}

\begin{equation}
B(t)=t^{3}P_{3}+3t^{2}(1-t)P_{2}+3(t(1-t)^{2})P_{1}+(1-t)^{3}P_{0}
\end{equation}

Onde, após cálculos se chegará a:

B(t)=  $\begin{bmatrix}
       t^{3} & t^{2} & t & 1          \\[0.3em]
		\end{bmatrix}$
		$\begin{bmatrix}
		       -1 & 3 & -3  & 1           \\[0.3em]
		        3 & -6 &  3 & 0   \\[0.3em]
		       -3 & 3 & 0 & 0 \\[0.3em]
		       1 & 0 & 0 & 0
		     \end{bmatrix}$
		$\begin{bmatrix}
		       P_{0}           \\[0.3em]
		       P_{1}   \\[0.3em]
		       P_{2} \\[0.3em]
		       P_{3}
		     \end{bmatrix}$


Quanto maior o número de pontos de controlo maior o custo computacional de as representar.

{\huge \textbf{Bezier Patches/superfície:}}

Se se tiver um 2D-array de pontos $P_{i,j}$,i=0,....,m,j=0,....n, então pode-se construir uma superfície bezier da mesma forma que se constrói uma curva Bezier. Da mesma maneira que nas curvas Bezier, a mais importante e mais usada é a superfície de Bezier bicúbica (m=n=3) onde é definida por 16 pontos de controlo $P_{i,j}$ e pode ser escrita da seguite forma:

\begin{equation}
B(u,v)=\sum_{j=0}^{3}\sum_{i=0}^{3}B_{i}(u)P_{i,j}B{j}(v)
\end{equation}



\begin{center}
 	
 	\includegraphics[scale=1,keepaspectratio]{resources/beziersupf.png}
 	\captionsetup{type=figure, width=0.8\linewidth}
	\caption{Superfície de Bezier bicúbica}
\label{fig:ssec1:diagram:plane:to:sphere} 
\end{center}

Sendo U = $\begin{bmatrix}
       u^{3} & u^{2} & u & 1          \\[0.3em]
		\end{bmatrix}$

e M = $\begin{bmatrix}
		       -1 & 3 & -3  & 1           \\[0.3em]
		        3 & -6 &  3 & 0   \\[0.3em]
		       -3 & 3 & 0 & 0 \\[0.3em]
		       1 & 0 & 0 & 0
		     \end{bmatrix}$

Com que foi previamente explicado relativamente às curvas de Bezier, conclui-se que o método de construção de uma superficie Bezier corresponde a um conjuntos de curvas de Bezier conjuntas. 
Então:

B(u,v) = $\begin{bmatrix}
       u^{3} & u^{2} & u & 1          \\[0.3em]
		\end{bmatrix}$ 
		M$\begin{bmatrix}
		       P_{00} & P_{01} & P_{02} & P_{03}   \\[0.3em]
		       P_{10} & P_{11} & P_{12} & P_{13}   \\[0.3em]
		       P_{20} & P_{21} & P_{22} & P_{23}   \\[0.3em]
		       P_{30} & P_{31} & P_{32} & P_{33}
		     \end{bmatrix}$
		$M^{T} \begin{bmatrix}
		       v^{3}           \\[0.3em]
		       v^{2}   \\[0.3em]
		       v^{1} \\[0.3em]
		       v^{0}
		     \end{bmatrix}$

Este método exato irá ser aplicado mais tarde nos algoritmos.