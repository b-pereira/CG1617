\documentclass[hidelinks,11pt,titlepage,a4paper]{article} %padrao letterpaper, 10pt

% ----------------------------------------------
%Letra parecida com Arial
\usepackage{helvet}
\renewcommand{\familydefault}{\sfdefault}
\usepackage[T1]{fontenc}
% ---------------------------------------------


\usepackage[utf8]{inputenc}
\usepackage[portuguese]{babel}
\usepackage{amsfonts,amssymb,graphicx}
\usepackage[centertags]{amsmath}
\usepackage[hmargin=2.5cm,vmargin=2cm,bmargin=3cm]{geometry}
\usepackage{fancyvrb}
\usepackage{graphicx}
\usepackage{eso-pic} % marca de água para logos
\usepackage{indentfirst}
\usepackage{mathtools}
\usepackage{caption}
\usepackage{cite}
\usepackage{url}
% clever reff %
\usepackage{hyperref}
\usepackage{amsmath}


\usepackage{eqparbox,array}
\usepackage{algorithm}
\usepackage[noend]{algpseudocode}

\makeatletter
\renewcommand{\ALG@name}{Algoritmo}
\renewcommand{\listalgorithmname}{Lista de \ALG@name s}
\makeatother

\usepackage{pdflscape}

\usepackage{cleveref}
% ----------- %

% ----------------------------------------------
\usepackage{booktabs}
% Tabelas fancy
% ----------------------------------------------

% ----------------------------------------------
%gráficos fancy
\usepackage{pgfplots}
%?
\usepackage{pgfplotstable}
% ----------------------------------------------

\usepackage{minted}
\usemintedstyle{emacs}
\setminted{
frame=lines,
framesep=2mm,
baselinestretch=1.2,
fontsize=\footnotesize,
%linenos, 
breaklines,
breakautoindent=false,
autogobble
}

%%%%%%%%%%%%  CABEÇALHO  %%%%%%%%%%%%%

\usepackage{fancyhdr}
\pagestyle{fancy}
\lhead{\includegraphics[height=0.53in]{resources/template/logo-ee.png}}
\rhead{\textsl{\leftmark}}
\cfoot{\thepage}
\renewcommand{\headrulewidth}{2pt}
\renewcommand{\footrulewidth}{1pt}

%%%%%%%%%%%%%%%%%%%%%%%%%%%%%%%%%%%%%%




%Usado em conjunto com o comando \layout. Este último mostra todos os valores
%dos elementos de página.
%\usepackage{layout}

%Mostra um conjunto de quadros/frames em cada página, correspondente ao tamanho
%de cada elemento de página, como rodapé, cabeçalho, intervalos entre estes,
%margem para notas
%
%\usepackage{showframe}

%Ajusta cabeçalho para altura do logotipo
\headheight= 44pt
%Coloca margem de notas dentro da página
\marginparwidth= 40pt


%%%%%%%%%%%%%%%%%%%%%%%%%%%%%%%%%%%%%%%%%%%%%%%%%%%%%%%%%%%%%%%%%%%%%%
%%USADO EM CONJUNTO COM O PACOTE LAYOUT
%%MOSTRA ELEMENTOS DA PÁGINA, COMO CABEÇALHO, RODAPÉ, MARGEM PARA NOTAS, CORPO
%%SEPARADORES ENTRE ELEMENTOS.
%%
%%
%\begin{figure}[htbp] 
%	\begin{center} \leavevmode 
%		\layout
%		\vspace{3cm}
%		\caption{Elementos da página. Os valores mostrados são aqueles que estão
%aplicados ao presente documento, não os valores por defeito } 
%\label{fig:layout} 
%	\end{center} 
%\end{figure}
%%%%%%%%%%%%%%%%%%%%%%%%%%%%%%%%%%%%%%%%%%%%%%%%%%%%%%%%%%%%%%%%%%%%%%%%

%--- Quotes ---%
%\usepackage{epigraph}
 
%\setlength\epigraphwidth{12cm} % default 8
%\setlength\epigraphrule{0pt}

%\begin{quote}
%	
%\end{quote}

\setcounter{secnumdepth}{4}
\setcounter{tocdepth}{4}
\setlength{\parindent}{2em}
\setlength{\parskip}{1em}
\renewcommand{\baselinestretch}{1.2}


%--- Custom Appendice ---%
%\usepackage[titletoc,toc,title]{appendix}

\pgfplotsset{compat=1.13} 



%--- Custom Appendice ---%
%\usepackage[titletoc,toc,title]{appendix}

\pgfplotsset{compat=1.13} 




\hypersetup{
pdftitle={Trabalho 1},
pdfauthor={Bruno Pereira},
%pdfsubject={Investigação Operacional},
%pdfkeywords={keyword1, keyword2}},
bookmarksnumbered=true,     
bookmarksopen=true,         
bookmarksopenlevel=1,       
colorlinks=true,            
pdfstartview=Fit,           
pdfpagemode=UseOutlines, % this is the option you were lookin for
pdfpagelayout=TwoPageRight
		}



\newenvironment{longlisting}{\captionsetup{type=listing}}{}
\begin{document}

%%%%%%%%%%%%%%%%%%%%%%%%%%%%%%%%%%%%%%%%%%%%

% ----------- %
\begin{titlepage}


\begin{minipage}{0.3\textwidth}
\begin{flushleft} 
\includegraphics[width=1.1\textwidth]{resources/template/logo-ee.png}
\end{flushleft}
\end{minipage}
\hfill
\begin{minipage}{0.6\textwidth}
\begin{flushright} 

\fontfamily{pag}\selectfont{\large \textsc{Universidade do Minho}\\[0.1cm]
\large \bfseries Mestrado Integrado em Engenharia Informática \\ [0.1cm]
\large \bfseries \textit{Computação Gráfica}\\[4mm]
}
\end{flushright}
\end{minipage}\\[1cm]


\vspace{3cm}


\begin{center}

\fontfamily{pag}\selectfont{\textsc{\Huge Trabalho Prático: 3º Fase}\\[1cm]


{\large \bfseries \Large\emph{Curvas, Superfícies Cúbicas e VBOs} \\[2cm] }



\vspace{3cm}

\begin{minipage}{0.7\textwidth}
\begin{flushright} \large
\textbf{Grupo 3:}\par
	Ricardo Silva  --- 60995\par
	Bruno Pereira  --- 72628	
\end{flushright}
\end{minipage}


\vfill

\emph{\large Braga, {\large \today}}
}
\end{center}

\end{titlepage}

% ----------- %
\tableofcontents
\newpage
\listoffigures
\newpage
% ----------- %


%---------------------------------------------------------------------------------------------------------------%
\section*{Introdução\markboth{\MakeUppercase{Introdução}}{}}
\addcontentsline{toc}{section}{Introdução}
\label{intro}


Para esta fase deste projecto é requerido o desenvolvimento de um novo tipo de modelo geométrico baseado em superfícies de Bezier, este novo modelo vai desempenhar o papel de um asterioide no sistema solar já desenvolvido na segunda fase. Este asteroide vai ter por si só uma órbita sua, consistente com a de um asteróide real, isto é, embora elíptica, a diferença entre a distância ao sol no momento da órbita de maior proximidade e a de maior distância irá ser maior do que qualquer planeta, dando assim um toque de realismo.

Adicionalmente, outra novidade irá ser as alterações relativas à translação e rotação dos planetas. As órbitas dos planetas agora irão ser sobre curvas Catmull-Rom, e as tranlações referentes a cada planeta deverão ser interpretadas do ficheiro XML em função de um ``time'' e dos seus pontos de controlo necessários para a curvatura Catmull-Rom. Quanto à rotação, nesta fase é subtituída a leitura pelo seu ``angle'' e passa ser igualmente por um ``time''. 

Finalmente, nesta fase todos os modelos apresentados irão ser desenhados com VBOs, em oposição ao modo imediato usado nas fases prévias.

Todas estas funcionalidades irão ser faladas em detalhe mais abaixo, dando ao leitor uma breve introdução teórica sobre cada tópico de modo a serem melhor compreendidas as decisões tomadas pelo grupo.
\clearpage


\section{Gerador}

O \texttt{Generator} tem, como função, recebendo parâmetros como comprimento,
largura, raio, etc., gerar ficheiros de texto com a extensão \verb|.3d|, cujo conteúdo
é a informação sobre as figuras a criar. 

Nesta secção ir-se-á descrever o processo de desenvolvimento das figuras
necessárias do sistema solar. As figuras pertinentes a desenhar são a esfera
(para os planetas e sol) e um disco (para alguns planetas que os tenham, como
por exemplo Saturno). 



\subsection{Esfera}
%---------------------------------------------------------------------------------------------------------------%
\subsubsection{Análise do Problema}
Para a construção da esfera teve-se que ter em conta coordenadas esféricas
modificadas para o referencial rodado com Y para cima, Z como eixo das abcissas
e X como eixo das ordenadas, como demonstra a Equação~\ref{eq:equ2}.


\begin{equation}
    \begin{cases}
    x = \cos(\phi) * \sin(\theta) * \rho \\
    y = \sin(\phi) * \rho \\
    z = \cos(\phi) * \cos(\theta) *\rho
    \end{cases}
\label{eq:equ2}
\end{equation}

Na Equação~\ref{eq:equ2}, $\rho$ representa o raio, $\phi$ o ângulo polar sendo
$\phi \in [-\dfrac{\pi}{2}, \dfrac{\pi}{2}]$, $\theta$ representa o ângulo
azimutal sendo $\theta \in [0, 2\pi]$. 


\newpage
\subsubsection{Diagrama}


\begin{center}
 	
 	\includegraphics[width=\textwidth,height=\textheight,keepaspectratio]{resources/sphere05.jpg}
 	\captionsetup{type=figure, width=0.8\linewidth}
	\caption{Objetivo do algoritmo de construção de esfera}
\label{fig:ssec1:diagram:plane:to:sphere} 
\end{center}



\begin{center}
 	
 	\includegraphics[keepaspectratio]{resources/esferaw.jpg}
 	\captionsetup{type=figure, width=0.8\linewidth}
	\caption{Diagrama de representativo de construção de esfera}
\label{fig:ssec1:diagram:sphere} 
\end{center}


No \emph{Figura~\ref{fig:ssec1:diagram:sphere}} pode-se ver uma matriz, que
representa a esfera nos graus de $\phi$ e $\theta$ para 6 \emph{stacks}
e 7 \emph{slices}. Assim como um mapa representativo da Terra, pretende-se
mostrar os pontos se a esfera fosse aplanada (ver
\emph{Figura~\ref{fig:ssec1:diagram:plane:to:sphere}}).

Em cada quadrícula são calculados 4 pontos iniciais, com base nos cálculos
apresentados pelo fórmula anterior. Note-se que, se usou duas variáveis para
guardar o $\phi$ anterior e o $\phi$ corrente, e $\theta$ anterior  e $\theta$
corrente. Adicionalmente é calculada a diferença de graus entre \emph{slices}
e \emph{stacks}, representados por $\Delta \phi$ e $\Delta \theta$,
respetivamente. 

A intenção é calcular cada quadricula para cada linha e coluna, com auxilio das
diferenças dos ângulos e à medida que se avança em cada quadricula, guardar
o último grau calculado ($\phi$ e $\theta$) e calcular nos pontos com
o incremento nestes ângulos. Assim desloca-se para a direita na matriz, conforme
$\theta$ avança de 0 para $2\pi$ e para baixo, conforme $\phi$ avança de
$\dfrac{\pi}{2}$ para $-\dfrac{\pi}{2}$ (sentido dos ponteiros do relógio).
O \emph{Algoritmo~\ref{alg:secc1:esfera}} representa este processo
e a \emph{Figura~\ref{fig:sec1:sphere:angles}} demonstra o que se
mencionou.  


\begin{center}
 	\includegraphics[width=\textwidth,height=0.5\textheight,keepaspectratio]{resources/esfera2.png}
 	\captionsetup{type=figure, width=0.8\linewidth}
	\caption{Diagrama de construção de esfera, com eixos, ordem do vértices
	e ângulos}
\label{fig:sec1:sphere:angles} 
\end{center}

O resultado pode-se ver na \emph{Figura~\ref{fig:ssec1:res:sphere}}, que
demonstra uma esfera em \emph{wireframe} gerada com a aplicação.  

\begin{center}
 	
 	\includegraphics[keepaspectratio]{resources/sphere.png}
 	\captionsetup{type=figure, width=0.8\linewidth}
	\caption{Esfera gerada}
\label{fig:ssec1:res:sphere} 
\end{center}
\newpage

\newgeometry{margin=1cm}
\begin{landscape}
\thispagestyle{empty} %% Remove header and footer.

\begin{algorithm}
\caption{Esfera}\label{alg:secc1:esfera}

\begin{center}
%\footnotesize %% Smaller font size.

\begin{algorithmic}[1]
\State$\Delta\_\theta \gets \dfrac{2\pi}{slices}$

\State$\Delta\_\phi \gets \dfrac{2\pi}{stacks}$

\State$prev\_\phi \gets \dfrac{\pi}{2}$

\State$current\_\phi \gets prev\_\phi - \Delta\_\phi$


\State$i \gets 0$
\While{$i \leq stacks$} 


\State$prev\_\theta \gets 0$
\State$current\_\theta \gets \Delta\_\theta$

\State$j \gets 0$

\While{$j \leq slices$} 


\State$Ponto A \gets raio*\cos(prev\_\phi) * \sin(prev\_\theta)$, $
raio*\sin(prev\_\phi)$, $raio*\cos(prev\_\phi) * \cos(prev\_\theta)$
\State$Ponto B \gets raio*\cos(current\_\phi)*\sin(prev\_\theta)$,$
 raio*\sin(current\_\phi)$,$
 raio*\cos(current\_\phi) * \cos(prev\_\theta)$
\State$Ponto C \gets raio*\cos(prev\_\phi) * 
 \sin(current\_\theta),raio*\sin(prev\_\phi)$,$
 raio*\cos(prev\_\phi) * \cos(current\_\theta)$
\State$Ponto D \gets raio*\cos(current\_\phi) *
 \sin(current\_\theta)$,$raio*\sin(current\_\phi)$,$raio*\cos(current\_\phi) *  \cos(current\_\theta)$
\newline
\State$Triangulo(Ponto A, Ponto B, Ponto D)$ \Comment{Guardado em ficheiro}
\State$Triangulo(Ponto B, Ponto C, Ponto D)$ \Comment{Guardado em ficheiro}
\newline
\State$prev\_\theta \gets current\_\theta$
\State$current\_\theta \gets current\_\theta + \Delta\_\theta $
\newline
\State$j \gets j + 1$ 

\EndWhile{}

\State$prev\_\phi \gets Current\_\phi$
\State$current\_\phi \gets Current\_\phi - \Delta\_\phi $
\State$i \gets i + 1$ 

\EndWhile{}

\end{algorithmic}

\end{center}

\end{algorithm}

\end{landscape}
\restoregeometry{}


\newpage


\subsection{Disco}
Nesta secção descreve os procedimentos usados para desenvolver um disco.
A motivação para o desenvolvimento desta figura provém da necessidade de
representar os anéis que rodeiam os planetas Saturno e Úrano.


\subsubsection{Análise do Problema}

Existem certos elementos do sistema solar, que são característicos de um modelo
do mesmo: anéis e órbitas. Apesar do significado de ambos ser diferente, ambos
podem ser desenhados com o mesmo objeto, variando apenas no raio interno
e externo.

Com efeito, requer-se para este projeto que se criem discos de vários tamanhos
para os anéis de Saturno e Neptuno, e para as órbitas de cada planeta. Note-se
que cada anel tem que ter alguma espessura, uma vez que, num plano, no
\emph{OpenGL} não se consegue ver o objeto. Assim cada disco terá duas
circunferências, uma interior e outra exterior, com raio interno e externo
respetivamente. Assim, as duas circunferências têm os mesmos pontos \emph{xx}
e \emph{zz} mas com uma distancia fixa no eixo \emph{yy}. 

A fórmula para desenhar uma circunferência está representada na
\emph{Equação~\ref{eq:equ3}} 

\begin{equation}
\begin{cases}
			x =  \sin(\theta) * r \\
	    z =  \cos(\theta) * r
\end{cases}
\label{eq:equ3}
\end{equation}



Nesta secção apresentam-se diagramas que explicam o processo de criação de uma
disco.

A \emph{Figura~\ref{fig:ssec1:disc}} representa a forma como a iteração será
feita, bem como apresenta de lado a espessura do disco.


\begin{center}
 	
 	\includegraphics[width=\textwidth,height=\textheight,keepaspectratio]{resources/disco.jpg}
 	\captionsetup{type=figure, width=0.8\linewidth}
	\caption{Diagrama Disco}
\label{fig:ssec1:disc} 
\end{center}

Como se pode verificar o diagrama é relativamente semelhante ao da esfera.
A matriz aqui observada apenas tem uma linha porque não se consideram
\textit{stacks} na representação do disco. As 8 colunas que representam as
8 \textit{slices} (estas 8 slides servem meramente para propósitos
exemplificativos). 


\begin{center}
 	\includegraphics[width=\textwidth,height=0.5\textheight,keepaspectratio]{resources/discodiagram.png}
 	\captionsetup{type=figure, width=0.8\linewidth}
	\caption{Pormenor dos vértices para desenho de um disco}
\label{fig:sec1:disc:vertex} 
\end{center}

Como se pode observar, o raciocínio é desenhar quadricula a quadricula com dois
triângulos cada, neste exemplo verifica-se que a primeira quadricula
é constituída pelos triângulos ABD e BCD.\ As coordenadas de cada ponto (vértice
dos triângulos) são calculados com o auxílio das variáveis angulares
prev\_$\theta $ e current\_$\theta$ usando a formulação das coordenadas
esféricas. A diferença entre esta é o comprimento/largura da quadricula que
corresponde a $\dfrac{2\pi}{slices}$. 

Ora este processo é referente à face superior do disco. Para desenhar a face
inferior faz-se o mesmo processo mas com outros vértices equivalentes nos eixos
\emph{xx} e \emph{zz} mas com uma diferença fixa de 0.010 \emph{yy} para
representar a altura.

Quanto à face lateral do disco o processo é idêntico ao representado na matriz
acima, mas enquanto que, para representar tanto a face superior como a inferior,
os pontos usados têm todos os o mesmo valor \emph{yy}, para representar o lado do disco
usa-se uma combinação dos pontos de ambas as faces. 

Este processo está representado no \emph{Algoritmo~\ref{alg:sec1:disco}},
e um resultado figura na \emph{Figura~\ref{fig:sec1:disc:res1}} e na
\emph{Figura~\ref{fig:sec1:disc:res2}}.  

\begin{center}
 	\includegraphics[width=\textwidth,height=0.5\textheight,keepaspectratio]{resources/disco1.png}
 	\captionsetup{type=figure, width=0.8\linewidth}
	\caption{Resultado de um disco em \emph{wireframe}, visto de baixo}
\label{fig:sec1:disc:res1} 
\end{center}


\begin{center}
 	\includegraphics[width=\textwidth,height=0.5\textheight,keepaspectratio]{resources/disco2.png}
 	\captionsetup{type=figure, width=0.8\linewidth}
	\caption{Resultado de um disco em \emph{wireframe}, visto de outro ângulo}
\label{fig:sec1:disc:res2} 
\end{center}




\newgeometry{margin=1cm}
\begin{landscape}
\thispagestyle{empty} %% Remove header and footer.
\begin{algorithm}
\caption{Disco}\label{alg:sec1:disco}

\begin{center}
%\footnotesize %% Smaller font size.

\begin{algorithmic}[1]
\State$\Delta\_\theta \gets \dfrac{2\pi}{slices}$


\State$prev\_\theta \gets 0$
\State$current\_\theta \gets \Delta\_\theta$

\State$i \gets 0$


\While{$i \leq slices$} 

\State$Ponto A \gets raioOut*\sin(prev\_\theta),
 0.005,
 raioOut*\cos(prev\_\theta)$

\State$Ponto B \gets raioIn*\sin(prev\_\theta),
 0.005,
 raioIn*\cos(prev\_\theta)$

\State$Ponto C \gets raioIn*\sin(current\_\theta),
 0.005,
 raioIn*\cos(current\_\theta)$

\State$Ponto D \gets raioOut*\sin(current\_\theta),
  0.005,
 raioOut*\cos(current\_\theta)$

\State$Ponto A2 \gets raioOut*\sin(prev\_\theta),
 -0.005,
 raioOut*\cos(prev\_\theta)$

\State$Ponto B2 \gets raioIn*\sin(prev\_\theta),
 -0.005,
 raioIn*\cos(prev\_\theta)$

\State$Ponto C2 \gets raioIn*\sin(current\_\theta),
 -0.005,
 raioIn*\cos(current\_\theta)$

\State$Ponto D2 \gets raioOut*\sin(current\_\theta),
  -0.005,
 raioOut*\cos(current\_\theta)$




\Comment{Lado de cima}

\State$Triangulo(Ponto D, Ponto B, Ponto A)$ \Comment{Guardado em ficheiro}
\State$Triangulo(Ponto C, Ponto B, Ponto D)$ \Comment{Guardado em ficheiro}



\Comment{Lado de baixo}
\State$Triangulo(Ponto A2, Ponto B2, Ponto D2)$ \Comment{Guardado em ficheiro}
\State$Triangulo(Ponto D2, Ponto B2, Ponto C2)$   \Comment{Guardado em ficheiro}
		  
\Comment{Lado de externo}
\State$Triangulo(Ponto A2, Ponto A, Ponto D2)$\Comment{Guardado em
ficheiro}
\State$Triangulo(Ponto D, Ponto D2, Ponto A)$\Comment{Guardado em
ficheiro}
  
\Comment{Lado de interno}
\State$Triangulo(Ponto B, Ponto C2, Ponto B2)$\Comment{Guardado em
ficheiro}
\State$Triangulo(Ponto C, Ponto C2, Ponto B)$\Comment{Guardado em
ficheiro}



\State$prev\_\theta \gets current\_\theta$
\State$current\_\theta \gets current\_\theta + \Delta\_\theta$

\State$i \gets i + 1$


\EndWhile{}
\end{algorithmic}
\end{center}

\end{algorithm}


\end{landscape}
\restoregeometry{}

\subsection{\emph{Patch} de Bézier baseado em ficheiro de pontos de controlo}

\subsubsection{Análise do Problema}

Para este projeto, requere-se que se obtenha pontos de controlo de uma
superfície de Bézier, através de uma ficheiro \emph{teapot.patch}, para
a construção de um bule de chá, para a representação de um cometa. O formato de
ficheiro é o que se segue:
\begin{enumerate}
	\item Número de \emph{patches};
	\item Índices para \emph{patches} (16 por linha, tantas linhas como nº de
		\emph{patches});
	\item Número de pontos de controlo;
	\item Ponto de controlo (coord.\ x, y, z) por linha, tantas linhas como nº de
		pontos de controlo;
\end{enumerate}

Para armazenar os dados do ficheiro, criou-se uma estrutura com dois valores
inteiros (nº \emph{patches} e nº de pontos), um vetor de vetores de inteiros,
para armazenar os valores dos índices e um vetor de \texttt{Point3d} para
armazenar os pontos de controlo. Além do mais, requere-se que se passe um valor
de tecelagem (\emph{tesselation}), para controlo da malha a criar.

A função de leitura separa o \emph{parsing} de cada elemento no formato descrito
atrás à custa de uma \emph{flag}. Note-se que o ficheiro original tinha os todos
os valores separados por vírgulas, pelo que, por uma questão de eficiência,
removeram-se as vírgulas, deixando os valores separados por espaços. Deste modo,
é possível utilizar as funcionalidades da linguagem de programação em uso
\texttt{C++} e saltar espaços a ler o ficheiro. Uma outra nota sobre o uso de
funcionalidades da linguagem, dado que se está a usar \texttt{vector} para
armazenamento de pontos e índices, o nº de \emph{patches} e o nº de pontos,
recebidos do ficheiro não são utilizados.  

Além do mais, por uma questão de legibilidade e facilidade de manutenção do
código, foram criados dois tipos de dados (\texttt{MatrixF} e \texttt{MatrixP},
escalares de vírgula flutuante e pontos 3D), criados às custas de vetores de
vetores, para guardar valores na forma matricial. Com efeito, uma
\texttt{MatrixP} guarda os pontos de controlo de uma superfície de Bézier
(matriz de pontos 4 $\times$ 4). 

A função \texttt{drawPatch}, trata do processamento de leitura de transformação
dos valores lidos em ficheiro num vetor de \texttt{MatrixP}, onde por último,
para cada matriz armazenada no vetor anterior, para um valor de $i$ e um valor
de $j$ dividido pela tecelagem, iterando cada um desses valor entre 0 e o valor
da tecelagem, cria-se 4 pontos da superfície de Bézier, para cada iteração, que
formem uma quadricula conforme mostra a \emph{Figura~\ref{}}, onde são retirados
os triângulos e armazenados num ficheiro \.3d, todos os vértices calculados.
Para calcular cada ponto da superfície de Bézier, é utilizada a função
\texttt{getBezierPatchPoint}, que recebe um valor de $u$, $v$ e uma matriz de
pontos de controlo. 

\paragraph{Curvas de Bézier}

Uma curva Bézier pode ser definida por um qualquer numero de pontos, pontos
estes chamados pontos de controlo da curva. Transformações como translação
e rotação podem ser aplicadas na curva manipulando estes pontos. 

O algoritmo de \emph{De Casteljeau} oferece uma forma de calcular uma curva,
baseada em 4 pontos de controlo, onde um parâmetro $t \in [0,1]$ varia, e um
ponto é obtido $t$ e $1-t$ entre cada reta de cada ponto de controlo. Cada ponto
intermédio (3 pontos em simultâneo, um em cada reta, para cada variação de $t$),
conecta com cada um dos 3 pontos anteriormente mencionados, criando duas retas
com $t$ a variar de igual modo, nas retas como nas retas dos pontos iniciais. Em
seguida, para cada variação de $t$ das duas retas, dois pontos intersetam-se, e,
por último, para cada variação de $t$, nesta última reta, temos um ponto da
curva de Bézier. Note-se que $t$ varia em simultâneo, para todas as retas
iniciais e intermédias. A \emph{Figura~\ref{fig:casteljeaut}},
e \emph{Figura~\ref{fig:casteljeautree}} ilustram esta explicação.  

\begin{center}	
 	\includegraphics[width=\textwidth,height=\textheight,keepaspectratio]{resources/casteljou.png}
 	\captionsetup{type=figure, width=0.8\linewidth}
	\caption{Algoritmo geométrico \emph{De Casteljeau}}
\label{fig:casteljeaut} 
\end{center}


\begin{center}	
 	\includegraphics[width=\textwidth,height=\textheight,keepaspectratio]{resources/casteljeau2.png}
 	\captionsetup{type=figure, width=0.8\linewidth}
	\caption{Representação da iteração em cada reta entre pontos referente
		à \emph{Figura~\ref{fig:casteljeaut}}}
\label{fig:casteljeautree} 
\end{center}

A partir da \emph{Figura~\ref{fig:casteljeautree}} é derivada
a \emph{Equação~\ref{eq:beziercurve}}.

\begin{equation}
B(t)=t^{3}P_{3}+3t^{2}(1-t)P_{2}+3t{(1-t)}^{2}P_{1}+{(1-t)}^{3}P_{0}
\label{eq:beziercurve}
\end{equation}

De uma forma mais condensada, a equação anterior pode ser representada como na
\emph{Equação~\ref{eq:beziercurvecondensed}} 


\begin{equation}
B(t)=\sum_{k=0}^{3}B_{3,k}(t)P_{k}
\label{eq:beziercurvecondensed}
\end{equation}


Note-se que a equação anterior é uma particularização de curva cúbica de Bézier,
podendo uma curva de Bézier, ter $n$ graus  onde n corresponde a ($\text{nº
pontos de controlo} - 1$), e $t \in [0,1]$ e o seu somatório é sempre igual a 1,
tomando a forma genérica da \emph{Equação~\ref{eq:beziercurvegenerica}}. 


\begin{equation}
B(t)=\sum_{k=0}^{n}B_{n,k}(t)P_{k}
\label{eq:beziercurvegenerica}
\end{equation}

\begin{center}
 	
 	\includegraphics[scale=0.5,keepaspectratio]{resources/exemplos1Bezier.png}
 	\captionsetup{type=figure, width=0.8\linewidth}
	\caption{Curvas exemplo de Bézier}
\label{fig:beziercubicexamples} 
\end{center}


A \emph{Equação~\ref{eq:beziercurve}} pode ser representada na forma matricial
segundo a \emph{Equação~\ref{eq:beziercurvematrix}}. 

\begin{equation}
B(t)= \begin{bmatrix}
       t^{3} & t^{2} & t & 1          
		\end{bmatrix}
		\begin{bmatrix}
		       -1 & 3 & -3  & 1 \\
		        3 & -6 &  3 & 0 \\
		       -3 & 3 & 0 & 0   \\
		        1 & 0 & 0 & 0
		\end{bmatrix}
		 \begin{bmatrix}
		       P_{0}   \\
		       P_{1}   \\
		       P_{2}   \\
		       P_{3}
		     \end{bmatrix}
\label{eq:beziercurvematrix}
\end{equation}

\newpage
\paragraph{Superfícies de Bézier (\emph{patches} de Bézier)}

Se se tiver um \emph{array} bi-dimensional de pontos
$P_{i,j}$, com $i\in[0, m]$, e $j\in[0,n]$, então pode-se
construir uma superfície Bézier da mesma forma usando um método similar a uma
curva cúbica de Bézier. Neste caso, ao invés de um parâmetro, existem dois
($u$ e $v$), onde $u\in[0,1]$ e $v\in[0,1]$.
Uma superfície de Bézier bi-cúbica (m=n=3), dado que tem por base curvas cúbicas
de Bézier definidas por 4 pontos de controlo, uma vez que é bi-dimensional, é definida por 16 pontos de
controlo $P_{i,j}$, e  representa-se pela \emph{Equação~\ref{eq:bezierpatch}}. Um
exemplo de uma superfície de Bézier está na \emph{Figura~\ref{fig:patchexample}} 

\begin{equation}
B(u,v)=\sum_{j=0}^{3}\sum_{i=0}^{3}B_{i}(u)P_{i,j}B{j}(v)
\label{eq:bezierpatch}
\end{equation}



\begin{center}
 	
 	\includegraphics[width=0.5\textwidth,height=0.5\textheight,keepaspectratio]{resources/beziersupf.png}
 	\captionsetup{type=figure, width=0.8\linewidth}
	\caption{Superfície de Bézier bi-cúbica}
\label{fig:patchexample} 
\end{center}
Temos que:
\begin{gather*}
M = \begin{bmatrix}
-1 &  3 & -3 & 1  \\
 3 & -6 & 3  & 0  \\
-3 &  3 & 0  & 0  \\
1  &  0 & 0  & 0  \\
\end{bmatrix}
\end{gather*}

O parâmetros $u$ e $v$ estão entre 0 e 1, à semelhança do parâmetro $t$ da
função para uma curva de Bézier, onde $M$ é a matriz dos coeficientes obtida da
\emph{Equação~\ref{eq:beziercurve}}. A explicação para a obtenção de um ponto
numa superfície de Bézier, e, grosso modo, análoga à obtenção de um ponto numa
curva de Bézier, sendo que neste caso, estão dois parâmetros a variar entre
0 e 1, criando uma malha de com curvas de Bézier, conforme está na
\emph{Figura~\ref{fig:patchexample}}. Derivando
a \emph{Equação~\ref{eq:bezierpatch}} obtém-se
a representação matricial na \emph{Equação~\ref{eq:bezierpatchmatrix}}.

\begin{equation}
B(u,v) = \begin{bmatrix}
       u^{4} & u^{2} & u & 1          \\
		\end{bmatrix}
		M\begin{bmatrix}
		       P_{00} & P_{01} & P_{02} & P_{03}   \\
		       P_{10} & P_{11} & P_{12} & P_{13}   \\
		       P_{20} & P_{21} & P_{22} & P_{23}   \\
		       P_{30} & P_{31} & P_{32} & P_{33}
		     \end{bmatrix}
		M^{T} \begin{bmatrix}
		       v^{3} \\
		       v^{2} \\
		       v^{1} \\
		       v^{0}
		     \end{bmatrix}
\label{eq:bezierpatchmatrix}				 
\end{equation}

\paragraph{Função \texttt{getBezierPatchPoint}}

Antes de descrever a função \texttt{getBezierPatchPoint} é necessário descrever
algumas funções cridas par utilizar nessa função. 

Para cálculos com valores escalares de vírgula flutuante, nomeadamente
multiplicação de matrizes, criou-se a função \texttt{multMatrix} que multiplica
duas matrizes. Para obter um cálculo correto, são obtidas as dimensões das
matrizes (nº de linhas e nº de colunas), tal que, as dimensões para uma dada
matriz M1 são m $\times$ n, e as dimensões para uma dada 
matriz M2 são p $\times$ q. Para obtenção da matriz resultado, tem-se em conta
o nº de linhas da matriz M1 (m) e o nº de colunas da matriz M2 (q), onde
a matriz resultante terá um dimensão m $\times$ q. Note-se que os valores n ---
nº de colunas da matriz M1 --- e p --- nº de linhas da matriz M2 têm que ser
iguais. No entanto, essa verificação não é feita no código, no entanto,
assume-se que o programador sabe da especificação desta função. Para guardar
o valor da multiplicação de matrizes (somatório da multiplicação das linhas com
as colunas), usa-se um acumulador de resultado, fazendo variar um índice k,
entre 0 e p (que poderia ser n).

A função \texttt{matrixPointToScalar} obtém as coordenadas de x, ou de y, ou de
z, da matriz de pontos de controlo e armazena esses valores numa matriz de
escalares.


A função \texttt{getBezierPatchPoint} utiliza
a \emph{Equação~\ref{eq:bezierpatchmatrix}}, sendo o código amigável na sua
leitura e interpretação, uma vez que abstrai detalhes de implementação através
dos tipos de matrizes já mencionados. Em primeiro lugar,  a função calcula
a matriz U e a matriz V, com os parâmetros $u$ e $v$ respetivos, conforme
a equação. A segunda parte do algoritmo é multiplicar a matriz U por M, e matriz
$M^T$ por V e guarda cada resultado numa matriz. Note-se que, $M = M^T$, então
a matriz M é reutilizada.

Para calcular o ponto da superfície de Bézier, é necessário obter cada
coordenada dos pontos de controlo, sendo que a matriz de escalares da coordenada
x será para calcular a coordenada x do ponto da superfície, a matriz de
escalares da coordenada y será para calcular a coordenada y do ponto da
superfície e  a matriz de escalares da coordenada z será para calcular
a coordenada z do ponto da superfície. Cada matriz de escalares é multiplicada
pelo resultado de $UM$, e o resultado desta, com o resultado de $M^{T}V$ ou $MV$.
As matrizes resultantes destas operações são matrizes com dimensão $1\times1$,
com cada valor da coordenada x, y e z do ponto da superfície. Essas coordenadas
são guardadas num \texttt{Point3d} que é retornado pela função. 

Os resultados da aplicação do algoritmo para uma tecelagem de 50 podem ser
vistos nas figuras abaixo.

\begin{center}	
 	\includegraphics[width=\textwidth,height=\textheight,keepaspectratio]{resources/teapotBaixo.png}
 	\captionsetup{type=figure, width=0.8\linewidth}
	\caption{Bule visto de baixo}
\label{fig:teapotbottom} 
\end{center}

\begin{center}	
 	\includegraphics[width=\textwidth,height=\textheight,keepaspectratio]{resources/teapotCima.png}
 	\captionsetup{type=figure, width=0.8\linewidth}
	\caption{Bule visto de cima}
\label{fig:teapotabove} 
\end{center}


\begin{center}	
 	\includegraphics[width=\textwidth,height=\textheight,keepaspectratio]{resources/teapotPontos}
 	\captionsetup{type=figure, width=0.8\linewidth}
	\caption{Pontos da superfície do bule}
\label{fig:teapotpoints} 
\end{center}




\clearpage

\section{Motor}

Nesta secção pretende-se descrever três pontos essenciais desta fase: as
estruturas de dados, algoritmos e processos de \emph{rendering}. Note-se que
o foco do \emph{rendering} será o uso de \emph{vertex array objects} por
oposição ao modo imediato de forma a obter uma maior eficiência.  


\subsection{\emph{Vertex Array Objects}}

O OpenGL possibilita dois modos de renderização: o modo imediato
e o \emph{vertex buffer objects} (VBOs). Os VBOs permitem um ganho substancial
de performance, uma vez que os dados são logo enviados para a memória da placa
gráfica onde residem, e por isso podem ser renderizados diretamente da placa
gráfica. O modo imediato usa a memória do sistema onde os dados são inseridos
\emph{frame} a \emph{frame}, usando uma API de renderização, o que causa peso
computacional sobre o processador.

Para a utilização das VBO's é necessário recorrer a criação de \emph{arrays} com os
dados para renderizar geometria. Com efeito, é necessário, em primeiro lugar,
ativar os \emph{arrays} com os diferentes tipos de dados e colocar os dados num
\emph{buffer object}. Estes \emph{arrays} são acedidos pelo seu endereço individual da
sua localização em memória, sendo então desenhadas as figuras geométricas dos
respetivos conteúdos dos \emph{arrays}.

Note-se que, no OpenGl, qualquer inteiro sem sinal pode ser usado como um
identificador de \emph{buffer objecto}. Estes identificadores podem-se
armazenados numa estrutura, sendo necessário, em seguida, gerar \emph{buffers} para os
vértices (um \emph{buffer} para cada \emph{array} com vértices ---
\texttt{glGenBuffers}) e ativar cada \emph{buffer} pelo seu identificador
(\texttt{glBindBuffer}) e preencher o \emph{buffer} com os dados de cada
\emph{array} de vértices previamente mencionados.

Para desenhar, a figura geométrica é necessário definir a semântica, ou seja,
definir o \emph{offset} relativo ao inicio do buffer (\texttt{glVertexPoint}),
consoante o tipo de dados, fazer o bind do objecto apropriado para fazer
a renderização e renderizar os arrays de vertices usando a função adequada
(\texttt{glDrawArrays} ou \texttt{glDrawElements}).

Para utilizar a tecnologia dos VBO's criou-se uma estrutura \texttt{Models}, que
para além de ser composta por um apontador para a raiz da árvore
\emph{n-ária} que será mencionada na próxima secção, possui um vetor de
\texttt{GLuint} para armazenar os identificadores e uma tabela com os nomes dos
ficheiros com os vértices para a geometria associadas a um tipo \texttt{VBO},
que é constituído pelo número de vértices do \emph{vertex array} e o índice do
\emph{vertex array} no vetor de identificadores.  

Para utilizar os VBO's, como descrito acima inicializou-se \emph{buffer}  com função
\texttt{initBuffers}, após da leitura do ficheiro de vértices,  ativando-o,
gerando um \emph{buffer} para um identificador, fazendo o \emph{bind} de
seguida e preenchendo o \emph{buffer} com os dados.   

Para renderizar os VBO's criou-se a função \texttt{drawVBO}, que faz
o \emph{bind} do \emph{buffer} pelo índice do \emph{buffer} no tipo
\texttt{VBO}, define a semântica, com o tamanho no tipo \texttt{VBO}.
A função \texttt{drawElement}, procura um nome de ficheiro na tabela mencionada
acima, e caso exista, aplica a função \texttt{drawVBO}.


\subsection{Estruturas de Dados para Transformações}
\label{subsec:sec2}

Como a estrutura do ficheiro XML é uma árvore \emph{n-ária} escolheu-se uma
estrutura que se seguiu a mesma lógica. Assim construiu-se um tipo de dados,
a partir de um \emph{struct} com vários campos para guardar os valores de cada
grupo, onde se encontra um vetor de apontadores para outras estruturas deste
tipo \texttt{Group}, como se pode ver na \emph{Figura~\ref{fig:ssec2:strut}}.
Além do mais, a estrutura base possui um vetor de \emph{strings} para guardar
os nomes de modelos que se encontrarem descritos no ficheiro XML.\ Existe também,
um outro vetor para guardar apontadores de objetos do tipo
\texttt{Transformation}, que representa de forma genérica qualquer transformação
geométrica.

\begin{center} 	
\includegraphics[width=\textwidth,height=\textheight,keepaspectratio]{resources/estrutura.png}
\captionsetup{type=figure, width=0.8\linewidth}
\caption{Árvore \emph{n-ária} para armazenamento de grupos}
\label{fig:ssec2:strut} 
\end{center}

Para a representação genérica de qualquer transformação geométrica, criou-se uma
estrutura de classes, onde a classe \texttt{Transformation} funciona como
superclasse abstrata, com cinco subclasses: \texttt{Rotation},
\texttt{Translation}, \texttt{Scale}, \texttt{AnimatedTranslation}
e \texttt{AnimatedRotation}. A superclasse possui um método virtual para
aplicação da transformação (\texttt{applyTrasformation}), que é aplicado no
contexto das suas subclasses quando invocado diretamente, ou como parte da
estrutura da hierarquia, à custa do polimorfismo que a linguagem de programação
C++ permite. Esta hierarquia pode ser vista com mais detalhe na
\emph{Figura~\ref{fig:ssec2:class}}. Parte da estrutura de classes é a classe
\texttt{Point3d} que representa um triplo de \texttt{floats} para as
coordenadas geométricas de um ponto em 3 dimensões. Adicionalmente a classe
implementa alguns métodos para cálculos de pontos e vetores, embora
o significado seja diferente entre estes dois conceitos. Algumas classes
implementam vetores deste tipo.

Posteriormente as classes \texttt{AnimatedTranslation}
e \texttt{AnimatedRotation} serão descritas com maior detalhe.



\begin{center} 	
\includegraphics[width=\textwidth,height=\textheight,keepaspectratio]{resources/classes.png}
\captionsetup{type=figure, width=0.8\linewidth}
\caption{Hieraquia de classes de transformações geométricas}
\label{fig:ssec2:class} 
\end{center}


\subsection{Descrição do processo de leitura}

A função principal de leitura é a que está demonstrada no
\emph{Algoritmo~\ref{alg:ssec2:leitura}} e representa parte do processo de
leitura, no entanto decidiu-se remover partes acessórias de inicialização de
estruturas e afins. Esta função serve-se da estrutura de apontadores da árvore
que compõem a estrutura do documento XML para navegar na estrutura
recursivamente. 

Para além do apontador para o estrutura XML, passou-se por parâmetro um
apontador para a estrutura \texttt{Models}, que contem um apontador para a raiz
da árvore \emph{n-ária} e uma tabela (ou mapa) com os valores dos instâncias do
tipo \texttt{VBO}  para \emph{rendering} com o nome do ficheiro associado
(chave).Note-se que as estruturas em que se guardam valores dos elementos fazem
todas parte de um \texttt{Group}, tanto como o vetor de \emph{strings} com
o valor dos ficheiro, bem como o vetor de transformações e outros \texttt{Group}
no vetor de grupo, o seu acesso é por uma apontador para \texttt{Group}, exceto
os pares chave/valor guardados numa tabela na estrutura \texttt{Modelos}, com
o acesso feito a estrutura por apontador para a mesma.  

\begin{algorithm}

\caption{Função Principal Leitura}
\label{alg:ssec2:leitura} 
\footnotesize %% Smaller font size.
\begin{algorithmic}[1]
\Procedure{readXMLFromRootElement}{$XMLElement * elem$, $Modelos * models$,
$Group * grupo$}

\If{$elem = NULL$} 

\Return{}

\EndIf{}

\ForAll{$elem \in SIBLINGS$} 

\If{$elem = TRANSLATION \lor elems = SCALE$} 



\If{$elem = TRANSLATION$} 


\If{$elem = Attribute(TIME)$} 

\Comment{Obter o atributo $time$ e os valor dos pontos de controlo para animação
da translação}
\Comment{os atributos $x$, $y$, $z$ são alternativos.Se aparecerem é lhes
atribuído um valor. Caso contrário ficam com o valor $0$}

\State{\texttt{Guardar $x$, $y$, $z$ numa transformação como translação animada
com órbita}} 
 
\Else{} 
  

\Comment{os atributos $x$, $y$, $z$ são alternativos.Se aparecerem é lhes
atribuído um valor. Caso contrário ficam com o valor $0$}

\State{\texttt{Guardar $x$, $y$, $z$ numa transformação como translação}} 

\EndIf{}
 
\Else{} 

\State{\texttt{Guardar $x$, $y$, $z$ numa transformação como escala}} 
  
\EndIf{}

\ElsIf{$elem = ROTATION$} 


\If{$elem = Attribute(TIME)$} 

\Comment{Obter o atributo $time$ e os atributos $axis_{x, y, z}$. Para estes
últimos, se aparecerem é lhes atribuído um valor. Caso contrário ficam com o valor $0$}


\Comment{os atributos $x$, $y$, $z$ são alternativos.Se aparecerem é lhes
atribuído um valor. Caso contrário ficam com o valor $0$}
\State{\texttt{Guardar $x$, $y$, $z$ numa transformação como rotação animada}} 
 
\Else{} 

\Comment{os atributos $axis_{x, y, z}$ e $angle$ são alternativos. Se aparecerem
é lhes atribuído um valor. Caso contrário ficam com o valor $0$}

\EndIf{}
\State{\texttt{Guardar $axis_{x, y, z}$ e $angle$ numa transformação como
rotação}}

\ElsIf{$elem = MODELS$} 
 
\ForAll{$model \in MODELS$} 

\State{\texttt{Guardar atributo $file$ no vetor de vector<\emph{string}>
apontado por $grupo$}} 

\State{\texttt{Carregar lista de triângulos num $map$ associado a $file$,
apontados por $models$}}

\EndFor{}

\ElsIf{$elem = GROUP$} 

\State{\texttt{Criar novo $* novo\_grupo$}}

\State{\Call{readXMLFromRootElement}{$elem$, $novo\_grupo$}} 
 
\EndIf{}

\EndFor{}

\EndProcedure{}

\end{algorithmic}
\end{algorithm}

Como se pode ver no \emph{Algoritmo~\ref{alg:ssec2:leitura}}, as primeira
instruções referem-se ao caso de paragem da função recursiva que verifica se
o apontador do elemento, representa é um apontador não nulo. Em caso de nulo,
a função retorna para o sítio de onde foi invocada. 
Em seguida itera-se cada elemento que esteja no mesmo nível da árvore do
documento XML.\ podendo o elemento representar uma rotação, uma escala, uma
translação, um modelos ou grupos de modelos, e por fim um grupo.

Com efeito, se o elemento encontrado representar uma escala ou uma
translação caso for um translação e encontrar um atributo \texttt{time}, lê
o valor nesse atributo e itera pelos elementos \texttt{point}, verificando os
atributos x, y e z de cada ponto adicionando ao vetor de pontos da translação
animada (\texttt{AnimatedTranslation}) e adiciona a transformação ao vetor de
transformações. Caso contráriovobtêm-se os atributos \emph{x}, \emph{y} e \emph{z} e cria-se um
apontador, alocando memória para um instância da classe \texttt{Translation}
conforme a sua existência.
Caso for uma rotação e encontrar um atributo \texttt{time}, lê
o valor nesse atributo e os os valores \emph{axisX}, \emph{axisY}
e \emph{axisZ}, e guarda uma \texttt{AnimatedRotation}.Caso contrário
cria um  \texttt{Rotation} com \emph{axisX}, \emph{axisY}, \emph{axisZ}
e \emph{angle}, conforme a sua existência. Em ambos os casos a transformação
é adicionada ao vetor de transformações.


Se o elemento representar um conjunto de modelos, é efetuada uma navegação por
apontador para todos os elementos desse conjunto, obtendo o valor do atributo
\emph{file}. Este é guardado no vetor de vetor de \emph{strings} de
\texttt{Group}. De igual modo, através do valor do ficheiro é feita uma leitura
do mesmo, que tem os valores dos pontos dos vértices dos triângulos para
\emph{rendering}. Note-se que cada linha do ficheiro representa um vértice,
dado que cada linha tem os valores dos vértices separados por espaço,
e cada vértice tem as coordenadas \emph{x}, \emph{y} e \emph{z}. 

Por último, caso seja encontrado um elemento grupo, podem ocorrer uma de duas
situações: se o grupo está ao mesmo nível do grupo que antecedeu, ou seja é um
elemento \emph{irmão} ou se está dentro de um grupo, isto é, é um \emph{filhos}
do grupo anterior. No entanto, note-se que não existe uma verificação dos dois
caso no algoritmo, sendo efetuada uma chamada recursiva, com um novo elemento
grupo. Para ilustrar o caso, atente-se na
\emph{Figura~\ref{fig:ssec2:recleitura}}. A primeira invocação deste função,
representada pelo algoritmo, é precedida pela inicialização de um
\texttt{Group}, sendo este passado como parâmetro para esta função. Ou seja
é a raiz da árvore de \texttt{Group} e não possui quaisquer valores. O primeiro
elemento que a função encontra é um grupo, como se pode ver na figura, logo tem
que ser inicializado e adicionado à raiz, e apontador deste novo \texttt{Group}
é passado por parâmetro para a chamada recursiva da função. Dentro da chamada
recursiva, vão sendo adicionadas as transformações e modelos e se houver um novo
grupo, o processo repete-se. Algo de salientar é que a raiz, não tem
transformações nem modelos nos respetivos, nunca, e apenas o vetor com os
apontadores para os filhos é que é preenchido. Dado que um elemento com
a \emph{tag} \texttt{scene} pode apenas ter grupos e não transformações, assim
se justifica a raiz não ter transformações. Para o caso da
\emph{Figura~\ref{fig:ssec2:recleitura}}, a raiz terá apenas um \emph{filho},
que será o \emph{pai} de todos os outros grupos.   

À medida que vão sendo encontrados elementos XML nulos, ou seja, que não há mais
nada no mesmo nível, a função ainda entra na chamada recursiva, mas logo
retorna. Para clarificar, o \texttt{Group} para qual foi criada memória, logo
antes desta chamada recursiva tem de existir e é uma folha. Além do mais, como
demonstra a figura, as sucessivas chamadas recursivas vão retornando e voltando
para o sítio onde forma invocadas. Nesta chamadas recursivas, como o apontador
para o \texttt{Group} pai permanece localmente nessa função, vão sendo
adicionados novos os \emph{irmãos}. 

\begin{center}
\includegraphics[scale=0.75,keepaspectratio]{resources/exampleXML.png}
\captionsetup{type=figure, width=0.8\linewidth}
\caption{Diagrama representativo da recursividade do processo de leitura}
\label{fig:ssec2:recleitura} 
\end{center}


\newpage
\subsection{Descrição do ciclo de \emph{rendering}}



Para fazer o \emph{rendering} da estrutura de dados em memória, implementou-se
uma função de travessia da árvore, colocada na função \texttt{renderScene}, após
a função \texttt{glLoadIndentity} e \texttt{gluLookAt}, nesta sequência.
O \emph{Algoritmo~\ref{alg:ssec2:traverse}} representa a função que efetua esta
travessia.

Com efeito, a primeira instrução é a \texttt{glPushMatrix}, uma vez que se
pretende colocar uma matriz para aplicação das transformações no topo da
\emph{stack} de matrizes do \emph{OpenGL}. Em seguida, para cada transformação
contida num \texttt{Group}, é invocada a função \texttt{applyTrasformation}, já
mencionada na \emph{Secção~\ref{subsec:sec2}}, que aplica as transformações
conforme o contexto, como já foi explicado.

Em seguida, é invocada a função \texttt{drawElement}, descrita no
\emph{Algoritmo~\ref{alg:ssec2:rendering}}. Esta função especifica a primitiva
para que será criada com os vértices em memória, neste caso com um triplo de
vértices. 


Os vértices estão contidos, em objetos do tipo \texttt{Triangle} que
por sua vez, contêm objetos do tipo \texttt{Point3d} com as coordenadas de cada
vértice. Para obter estes vértices e criar a primitiva com \texttt{glVertex3f},
é necessário obtêm todas as \emph{strings} com o nome dos ficheiros no vetor de
\emph{string} de cada \texttt{Group}. Como cada nome, procura-se pelo
nome de ficheiro na tabela a partir do apontador para \texttt{Modelos}. Se
a entrada existir, obtêm-se todos valores de \texttt{Triangle}, respetivos
vértices e coordenadas.

No seguimento desta instrução existe um ciclo para aceder a elementos do vetor
de apontadores pelo índice para \texttt{Group}. Cada elemento (apontador para
\texttt{Group}) em determinada posição do \emph{array} é passado por argumento
para a chamada recursiva da função. Quando o chamada recursiva retorna, o índice
é incrementado e o processo repete-se.\ A função termina quando não houver mais
elementos no vetor.  Ou seja, o índice é incrementando à medida que
a chamadas recursivas entram na memória automática (\emph{stack}) e retornam,
saindo da \emph{stack}. 

Por último, note-se que a \texttt{glPopMatrix} é executada logo após o retorno
da chamada recursiva. Uma vez que as transformações sejam herdadas é necessário
colocar matrizes na \emph{stack} de matrizes do \emph{OpenGL}, conforme se vai
descendo na árvore. Quando a função faz o \emph{pop} à \emph{stack} de matrizes,
faz-lo na mesma chamada recursiva da função.

\newpage

\begin{algorithm}
\caption{Função de travessia da árvore de \texttt{Group}}
\label{alg:ssec2:traverse} 
\footnotesize %% Smaller font size.
\begin{algorithmic}[1]

\Procedure{traverseTree}{$Modelos * models$, $Group *grupo$}

\State{\Call{glPushMatrix}{$~$}} 

\ForAll{$transformation \in grupo \to TRANSFORMATIONS$} 
 
\State{$transformation\to \Call{applyTransformation}{ }$} 
 
\EndFor{}

\State{\Call{drawElement}{$models$, $grupo$}} 

\State{$i \gets 0$} 

\While{$i < $ tamanho do \emph{array} $grupo \to FILHOS$}

\State{\Call{traverseTree}{$models$, $grupo\to FILHOS_{i}$}} 

\State{\Call{glPopMatrix}{$~$}} 

\State{$i \gets i = i + 1$}

\EndWhile{}

\EndProcedure{}

\end{algorithmic}
\end{algorithm}
%%%%%%%%%%%%%%%%%%%%%%%%%%%%%%%%%%%%%%%%%%%%%%%%%%%%%%%%%%

\subsection{Classe \texttt{AnimatedRotation}}

A classe \texttt{AnimatedRotation}, no método \texttt{applyTrasformation}
divide 360 graus pelo valor de \texttt{time} obtendo um $\Delta_{\text{grau}}$.
Em seguida é obtido o tempo global da aplicação com
\texttt{glutGet (GLUT\_ELAPSED\_TIME)}, sendo calculado um valor de $t$, através
do módulo do tempo global da aplicação, com a variável \texttt{time}*1000 (este
calculo é para apresentar o time nas mesma unidades do tempo global, ou seja
ms), sendo que o valor do módulo é divido por \texttt{time}*1000 para obter um
valor entre 0 e 1. O tempo global funciona como contador global, e quando
dividido pelo tempo da animação torna-se circular.

Para aplicar a animação de rotação basta multiplicar $t$ por 360 para obter um
valor entre 0 e 360.


\subsection{Classe \texttt{AnimatedTranslation}}

A classe \texttt{AnimatedTranslation} aplica o mesmo conceito do parâmetro
$t$ como descrito na secção anterior, para obter um valor entre 0 e 1. Note-se
que esta classe possui um vetor com os pontos de controlo para para curva cúbica
de \emph{Catmull-Rom}. Note-se que são no mínimo são 4 pontos, por que é o que
está na definição da curva, podendo ser mais, mas sempre iterando de 4 em
4 pontos (3 pontos anteriores mais um novo ponto do vetor).

O método \texttt{applyTrasformation} aplica a função
\texttt{renderCatmullRomCurve} e \texttt{getGlobalCatmullRomPoint} fazendo
a translação do ponto obtido desta.   

\paragraph{\emph{Splines} cúbicas \emph{Catmull-Rom}}

A curva cúbica \emph{Catmull-Rom} para dados pontos de controlo $P_{0}, P_{1},
P_{2} e P_{3}$, está definida de modo a que a tangente em cada ponto $P_{i}$
possa ser encontrada através da diferença entre os seus pontos vizinhos
$P_{i-1}$ e $P_{i+1}$

\begin{center}
 	
 	\includegraphics[scale=0.5,keepaspectratio]{resources/catmullDeriv.png}
 	\captionsetup{type=figure, width=0.8\linewidth}
	\caption{\emph{Spline} cúbica Catmull-Rom para os pontos $P_{0}, P_{1}, P_{2} e P_{3}$}
\label{fig:ssec1:diagram:plane:to:sphere} 
\end{center}

Esta curva pode ser escrita em forma de matriz:

\begin{gather*}
\frac{P_{2}-P_{0}}{2} = P_{1}^{'} = P^{'}(0) = c  \\
P_{1}                 = P(0)      = d        		  \\
P_{2}                 =P(1)       =a+b+c+d    		\\
\frac{P_{3}-P_{1}}{2} = P_{2}^{'} = P^{'}(1) = 3a+2b+c 
\end{gather*}


O que é equivalente a:

\begin{gather*}
P_{0} = a+b-c+d 		\\
P_{1} = P{0} = d 		\\
P_{2}=P(1)=a+b+c+d 	\\
P_{3}=6a+4b+2c+1
\end{gather*}

Este conjunto de equações pode-se representar na seguinte operação de matrizes:
\begin{equation}	
P=\begin{bmatrix}
		       P_{0} \\
		       P_{1} \\
		       P_{2} \\
		       P_{3}
		     \end{bmatrix}
= \begin{bmatrix}
 1 & 1 & -1 & 1 \\
 0 & 0 & 0 & 1 \\
 1 & 1 & 1 & 1 \\
 6 & 4 & 2 & 1
\end{bmatrix}
\begin{bmatrix}
a_{x} & a_{y} & a_{z}    \\
b_{x} & b_{y} & b_{z}    \\
c_{x} & c_{y} & c_{z}    \\
d_{x} & d_{y} & d_{z} 
\end{bmatrix} = C \times A
\label{eq:catmullmatrix}
\end{equation}
\begin{equation}
A = C^{-1}P
\label{eq:catmullmatrixa}
\end{equation}

Estes cálculos até agora demonstrados irão ser aplicados algoritmicamente no programa da seguinte maneira:
\begin{equation}	
\begin{bmatrix}
       x(u) & y(u) & z(u)             \\
\end{bmatrix} = 
\begin{bmatrix}
       t^{3} & t^{2} & t & 1          \\
\end{bmatrix}
\begin{bmatrix}
-0.5 & 1.5 & -1.5 & 0.5 \\
1 & -2.5 & 2 & -0.5     \\
-0.5 & 0 & 0.5 & 0      \\
0 & 1 & 0 & 0
\end{bmatrix}
\begin{bmatrix}
P_{0} \\
P_{1} \\
P_{2} \\
P_{3}
\end{bmatrix}
\label{eq:formula}
\end{equation}



\paragraph{\texttt{getGlobalCatmullRomPoint}}

Esta função aplica a fórmula da \emph{Equação~\ref{eq:formula}}, obtendo os
4 pontos de controlo, com os vetores com as coordenadas de forma similar ao
descrito nos \emph{patches} de Bézier  

\paragraph{\texttt{renderCatmullRomCurve}}

Itera de forma circular sobre os pontos armazenados no vetor de pontos com um
$t$ global obtendo todos os pontos da curva de Catmull-Rom





\clearpage

\section{Resultados}


Para criar o modelo do sistema solar tomaram-se algumas liberdades em relação
à distância dos planetas ao Sol, e como tal os valores das translações são mais
ou menos obtidos por experimentação. Não obstante,  os planetas estão colocados
sobre o eixo dos \emph{xx}. Para as órbitas, teve-se que ter em conta
o raio interior do anel (como o anel é bastante fino, é redundante calcular
o valor médio), e fazer uma escala ao anéis, é preciso dividir a distância
à origem, pelo valor do raio interior do anel. As rotações tiveram como base
valores reais, sendo o eixo de rotação o vetor (0,0,1).


Os resultados obtidos estão nas imagens seguintes.


A \emph{Figura~\ref{fig:ssec3:ptl}} mostra o grupo de planetas terrestres e lua
da Terra. 

\begin{center}
 	
 	\includegraphics[width=\textwidth,height=\textheight,keepaspectratio]{resources/pormenorTerrestres.png}
 	\captionsetup{type=figure, width=0.8\linewidth}
	\caption{\textit{Rendering} do modelo com foco do grupo de planetas terrestres e Lua}
\label{fig:ssec3:ptl} 
\end{center}


A \emph{Figura~\ref{fig:ssec3:tilt}} mostra Urano e Neptuno, com o \emph{tilt}
axial aplicado. 

\begin{center}
 	
 	\includegraphics[width=\textwidth,height=\textheight,keepaspectratio]{resources/pormenorTiltAxial.png}
 	\captionsetup{type=figure, width=0.8\linewidth}
	\caption{\textit{Rendering} do modelo com foco no \textit{tilt} axial de Urano e Neptuno}
\label{fig:ssec3:tilt} 
\end{center}


A \emph{Figura~\ref{fig:ssec3:modelo}} mostra o modelo completo. 

\begin{center}
 	
 	\includegraphics[width=\textwidth,height=\textheight,keepaspectratio]{resources/modelo.png}
 	\captionsetup{type=figure, width=0.8\linewidth}
	\caption{\textit{Rendering} do modelo final}
\label{fig:ssec3:modelo} 
\end{center}

\clearpage



%---------------------------------------------------------------------------------------------------------------%
\section*{Conclusão\markboth{\MakeUppercase{Conclusão}}{}}
\addcontentsline{toc}{section}{Conclusão}
\label{concl}

Em suma, podemos concluir que o projeto foi um sucesso parcial, dado que não foi
possível iluminar os topos do bule, e que possivelmente poderiam ter sido
efetuados mais testes. Não foram geradas coordenadas de textura para a caixa,
cone e plano, nem vetores normais destas geometrias. No entanto, ficaram
implementadas os vários tipos de luzes, com as suas propriedades de cor, bem
como as propriedades materiais da geometria, assim como a aplicação de texturas.


Como trabalho futuro, sugere-se implementar o cálculo dos vetores normais
e coordenadas de textura para as geometrias que faltam, bem tentar o cálculo de
vetores normais através de interpolação para o bule, tentando diminuir o tamanho
de cada triângulo, para introduzir o efeito MACH ou usar diretamente o modelo
\emph{flat} sabendo das desvantagens de quaisquer um destes métodos. 



\clearpage
% ----------- %
%\bibauthoryear
\nocite{*}
\bibliography{references}
\bibliographystyle{IEEEtran}
%\bibliographystyle{ieeetr}
\clearpage
%%%%%%%%%%%%%%%%%%%%%%%%%%%%%%%%%%%%%%%%%%%%



\appendix 

%\part*{ANEXOS}
%\addcontentsline{toc}{section}{ANEXOS}
\appendix

\part*{ANEXOS}
\addcontentsline{toc}{part}{ANEXOS}
\refstepcounter{part} 

\section{Modelo do Sistema Solar}
\label{appendix:a}
\begin{longlisting}
	\inputminted{xml}{resources/sistema.xml}
	\caption{Código XML com parãmetros para o sistema solar}
\label{listing:b}
\end{longlisting}


\end{document}

%%%%%%%%%%%%%%%%%%%%%%%%%%%%%%%%%%%%%%%%%%%%
